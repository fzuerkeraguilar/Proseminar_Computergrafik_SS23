\documentclass{thesisclass}
% Based on thesisclass.cls of Timo Rohrberg, 2009
% ----------------------------------------------------------------
% Thesis - Main document
% ----------------------------------------------------------------
% No empty in-between chapters, remove this line if desired for double-page printing
\let\cleardoublepage\clearpage

% you can add custom packages that you may need
\usepackage{amsmath}

%% -------------------------------
%% |  Information for PDF file   |
%% -------------------------------
\hypersetup{
 pdfauthor={Fabian Zürker Aguilar},
 pdftitle={Simulation of camera artefacts in real-time applications},
 pdfsubject={Not set},
 pdfkeywords={Not set}
}


%% ---------------------------------
%% | Information about the thesis  |
%% ---------------------------------

\newcommand{\myname}{Fabian Zürker Aguilar}
\newcommand{\mytitle}{Simulation of camera artefacts in real-time applications}
\newcommand{\myinstitute}{Institut für Visualisierung und Datenanalyse,\\ Lehrstuhl für Computergrafik}

\newcommand{\reporttype}{Proseminar}                                  % Proseminar-/Seminar-/Bachelor-/Master
%\newcommand{\reviewerone}{Prof. Dr.-Ing. Carsten Dachsbacher}     % only for bachelor/master
%\newcommand{\reviewertwo}{Prof. Dr. Hartmut Prautzsch}            % only for bachelor/master
\newcommand{\advisorone}{Max Piochowiak}
%\newcommand{\advisortwo}{?}                                      % only if applicable

\newcommand{\timestart}{XX. Monat 20XX}
\newcommand{\timeend}{XX. Monat 20XX}
\newcommand{\submissiontime}{DD. MM. 20XX}


%% ---------------------------------
%% | ToDo Marker - only for draft! |
%% ---------------------------------
% Remove this section for final version!
\setlength{\marginparwidth}{20mm}

\newcommand{\margtodo}
{\marginpar{\textbf{\textcolor{red}{ToDo}}}{}}

\newcommand{\todo}[1]
{{\textbf{\textcolor{red}{(\margtodo{}#1)}}}{}}


%% --------------------------------
%% | Old Marker - only for draft! |
%% --------------------------------
% Remove this section for final version!
\newenvironment{deprecated}
{\begin{color}{gray}}
{\end{color}}


%% --------------------------------
%% | Settings for word separation |
%% --------------------------------
% Help for separation:
% In german  the following hints are additionally available:
% "- = Additional separation
% "| = Suppress ligation and possible separation (e.g. Schaf"|fell)
% "~ = Hyphenation without separation (e.g. bergauf und "~ab)
% "= = Hyphenation with separation before and after
% "" = Separation without a hyphenation (e.g. und/""oder)

% Describe separation hints here:
\hyphenation{
% Pro-to-koll-in-stan-zen
% Ma-na-ge-ment  Netz-werk-ele-men-ten
% Netz-werk Netz-werk-re-ser-vie-rung
% Netz-werk-adap-ter Fein-ju-stier-ung
% Da-ten-strom-spe-zi-fi-ka-tion Pa-ket-rumpf
% Kon-troll-in-stanz
}


%% ------------------------
%% |    Including files   |
%% ------------------------
% Only files listed here will be included!
% Userful command for partially translating the document (for bug-fixing e.g.)
\includeonly{%
titlepage,
content
}


%%%%%%%%%%%%%%%%%%%%%%%%%%%%%%%%%
%% Here, main documents begins %%
%%%%%%%%%%%%%%%%%%%%%%%%%%%%%%%%%
\begin{document}

% Remove the following line for German text
%\selectlanguage{ngerman}
\selectlanguage{english}

\frontmatter
\pagenumbering{roman}
\include{titlepage}
\blankpage

\chapter*{Abstract}
With the rapid advance in computer graphics in various application like video games, virtual reality and augmented reality, the generation of immersive real-time images has become a critical component of these future systems.
To increase immersiveness of the application it is critical to also recreate the flaws of the human eye or camera systems.
However, a realistic simulation of the optical system of a camera is currently outside the scope of real-time images.
We explore a range of techniques to approximate artifacts of a camera system that are expected by a viewer in cinematic contexts.
These techniques are applicable in a wide range of application ranging from mobile virtual reality to video games played on a computer or console.

%% -------------------
%% |   Directories   |
%% -------------------
\tableofcontents
\blankpage


%% -----------------
%% |   Main part   |
%% -----------------
\mainmatter
\pagenumbering{arabic}
%% content.tex
%%

%% ==============================
\chapter{Introduction}
\label{ch:Introduction}


%% ==============
\chapter{Background on camera systems}
\label{ch:Content1}


\section{Lens systems}

\dots

\section{Aperture and sensor}

\dots

%% content.tex
%%

%% ==============
\chapter{Lens artifacts}
\label{ch:Content2}
%% ==============
\dots
%% ===========================
\section{Chromatic aberration}
\label{ch:Content2:sec:Section1}
%% ===========================

\dots

%% ===========================
\section{Lens flare}
\label{ch:Content2:sec:Section2}
%% ===========================

\dots

\chapter{Depth-of-Field}

\chapter{Motion Blur}

\chapter{Conclusion and future work}



%% --------------------
%% |   Bibliography   |
%% --------------------
\cleardoublepage
\phantomsection
\addcontentsline{toc}{chapter}{\bibname}

\iflanguage{english}
{\bibliographystyle{IEEEtranSA}}	% english style
{\bibliographystyle{babalpha-fl}}	% german style
												  
% Use IEEEtran for numeric references
%\bibliographystyle{IEEEtranSA})

\bibliography{ausarbeitung}
%\Erklaerung
\end{document}
