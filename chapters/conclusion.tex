\chapter{Conclusion and future work}
This paper presents a variety of techniques to simulate both motion blur and \gls{dof} in real-time applications.
Even though these techniques are approximations, they offer a considerable improvement to the realism and artistic control over the generated images.

We described the underlying physical phenomena and how each method approximates tries to approximate them.
Our paper compares these methods, evaluating them based on their performance and the quality of the images they produce.
As highlighted throughout the paper, there always exists a trade-off between between image accuracy and computational performance.
In real-time applications simpler methods often reach an acceptable image quality, as each image is only displayed for a fraction of time.
However, there is a continual pursuit for methods that can deliver higher fidelity without compromising the real-time aspect.

Looking towards the future, the proliferation of real-time ray-tracing hardware and supporting applications offers new opportunities for more realistic visual effects.
The real-time simulation of light rays, has the potential to dramatically enhance the depiction of depth of field and motion blur.
Although traditionally a resource-intensive process, the advent of dedicated ray-tracing hardware makes it increasingly feasible for real-time scenarios.
