\chapter{Motion blur}
In most real-world applications neither the camera or the scene are static.
An instantaneous snapshot of the scene would show none of this motion.
As such high speed motion in the scene or of the camera cannot be accurately represented without increasing the frame rate.
This causes temporal aliasing which is jarring and disorienting for the viewer.
This is exaggerated in interactive application as there is no control of the camera.
As such there are multiple approaches to reduce temporal aliasing in application where computation resources and thus frame rates are limited.

\section{Geometry substitution}
Instead of alleviating temporal aliasing by increasing the frame rate, one can modify the underlying geometry to approximate a moving object.
The geometry and its texture are "smeared" across two time steps.

\section{Accumulation buffers}
For each frame render scene n times with small time steps and average images.
Very high cost.
Converges on true solution

Keep last n frames in accumulation buffer.
When next frame is ready add with alpha to accumulation buffer and subtract the oldest frame.
Oldest frame needs to be re-rendered.
Increases input lag. Closest to realistic rendering. Roughly doubles the workload.

\section{Motion fields}
Store for each pixel it's motion relative to the camera. 
