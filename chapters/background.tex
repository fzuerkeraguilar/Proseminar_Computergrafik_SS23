\chapter{Background on camera systems}
\label{ch:background}

\section{Lens systems}
The simplest camera model is the pinhole camera.
An infinitesimal small hole is punched into an opaque material and creates the aperture.
An image is then taken from an image plane places at distance $d$ parallel to the aperture.
This camera model can be implemented easily in computer graphics software by modeling the aperture as a singe point in space.
The incoming light is then approximated by a ray emanating at an object or light source and crossing the aperture point.
This ray then intersects the image plane of the virtual camera at a single point and a color value is recorded.
This procedure is typically reversed for actual graphics pipelines with rays coming emanating at predetermined points on the image plane and crossing the aperture point.

If the aperture of a pinhole camera is assumed to be an infinitesimally small hole i.e. a point in space and light is approximated by straight rays, then the pinhole camera represents the ideal perspective camera.
Each point on the image plane only receives light from exactly one point and is therefore perfectly in focus.
The camera can therefore be described as a coordinate transformation $x_c, y_c, z_c \mapsto x, y$.
The resulting image coordinates $x, y$ can be derived from the equations:
\begin{align}
    \frac{x_c}{x} = -\frac{z_c}{d} \\
    \frac{y_c}{y} = -\frac{z_c}{d}
\end{align}
This results in the following projection:
\begin{align}
    \begin{pmatrix}
    x_c \\
    y_c \\
    z_c
\end{pmatrix} 
\mapsto
\begin{pmatrix}
    x \\
    y
\end{pmatrix}
= -\frac{d}{z_c}
\begin{pmatrix}
    x_c \\
    y_c
\end{pmatrix}
\end{align}
The negative sign represents the flipping of the image on the image plane.

In reality this camera model suffers from multiple issues that make pinhole cameras unpractical for photography.
The most obvious problem is the impossibility to create a infinitesimally small hole.
As such the hole will always have a real diameter $D$.
This causes a point on the image plane to receive light from multiple sources and thus negates the theoretically perfect focus of the pinhole camera.
If one still tries to approach a $D$ of $0$, one will notice that the light entering the pinhole will also approach $0$.
The image will become darker or the exposure time needed for a bright image will approach infinity.
Additionally at such small values of $D$ the assumption of light as a ray will break down and one will notice diffraction start to appear and blurring the image further.

To increase the amount of light hitting the image plane one or more lenses are used in real camera systems.
Lenses are used to bundle all light coming from a single direction which hits the lens to a single point on the image plane.


\section{Aperture and sensor}


\dots