%% ==============
\chapter{Lens artifacts}
\label{ch:Content2}
%% ==============
Lens artifacts are added 
%% ===========================
\section{Chromatic aberration}
\label{ch:Content2:sec:Section1}
%% ===========================
The rendered frame is split into it's color channels.
Each image is distorted radially by a specific amount.

%% ===========================
\section{Lens flare}
\label{ch:Content2:sec:Section2}
%% ===========================
Typically added to indicate high brightness to viewer as most monitors do not support a high enough dynamic range to display bright scenes correctly.
Lens flare is a fixed transparent texture that is blended with computed frame.
The image position of a light source is mirrored at the middle of the frame and the texture is placed there.
The size and opaqueness of the texture increase as they move toward the center of the frame.